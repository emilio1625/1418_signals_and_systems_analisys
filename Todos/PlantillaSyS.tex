\documentclass[10pt,a4paper]{report}
\usepackage[utf8]{inputenc}
\usepackage[spanish]{babel}
\usepackage{amsmath}
\usepackage{amsfonts}
\usepackage{amssymb}
\usepackage{graphicx}
\usepackage{fancyhdr}
\usepackage{float}
\usepackage{color}
\usepackage{mcode}
\usepackage[left=2cm,right=2cm,top=2cm,bottom=2cm]{geometry}
%%%%%%%%%
%%Incluir No. de equipo, integrantes y tema asignados
\author{Miembro1\\Miembro2\\Miembro3\\Equipo 20}
\title{Tema asignado}


\pagestyle{fancy}
\fancyhf{}
\lhead[]{}
\chead[]{}
\rfoot{\thepage}
\lfoot[]{}
\cfoot[]{}

\renewcommand\headrule
{{\color[RGB]{98,36,35}%
		\hrule height 2pt
		width\headwidth
		\vspace{1.3pt}%
		\hrule height 1pt
		width\headwidth
	}}
	\addto\captionsspanish{\def\tablename{Tabla}}%imprime Tabla en lugar de Cuadro
	%%
	\spanishdecimal{.}


\begin{document}
\thispagestyle{fancy}
\maketitle
\tableofcontents
\newpage
\setcounter{chapter}{1}
\section[Introducción]{Introducción}

\begin{equation}
\int_{t_{1}}^{t_{2}}\delta(t)dt =
\left\{
\begin{aligned}
1\quad t_{1}<0<t_{2}\\
0\quad otro \:valor
\end{aligned}
\right.
\end{equation}
%%
%%
\[
\delta(t)=0\qquad t\neq0
\]
\subsection{Escalón Unitario}
Escalón Unitario(Ec. \ref{escunit}) $u(t), u_{-1}(t)$\\
%%
\begin{equation}
 u(t)=
 \left\{
  \begin{aligned}
   1\quad t>0\\
   0\quad t<0\
  \end{aligned}
 \right.
\label{escunit}
\end{equation}
%%
\subsection{Rampa}
Rampa(Ec. \ref{rampa}) $ramp(t)$, $r(t)$, $u_{-2}(t)$\\

\begin{equation}
ramp(t)=
\left\{
\begin{aligned}
t\quad t\geq0\\
0\quad t<0\
\end{aligned}
\right.
\left\}=
\begin{aligned}
\int_{-\infty}^{t}u(\lambda)d\lambda=tu(t)
\end{aligned}
\right.
\label{rampa}
\end{equation}

\subsection{Signo}
Función Signum(Ec. \ref{signo}) $sgn(t)$\\

\begin{equation}
sgn(t)=
\left\{
\begin{aligned}
1\quad t>0\\
0\quad t<0\
\end{aligned}
\right.
\left\}=
\begin{aligned}
2u(t)-1
\end{aligned}
\right.
\label{signo}
\end{equation}

%
\subsection{Triangulo}
Función Triángulo(Ec. \ref{triang}) $tri(t)$

\begin{equation}
tri(t)=
\left\{
\begin{aligned}
1-|t|\quad |t|<1\\
0\qquad\:|t|\geq0\
\end{aligned}
\right.
\label{triang}
\end{equation}



%
\subsection{Señal Sinc}
Función $sinc(t)$ (Ec. \ref{sinc})\\
\begin{equation}
sinc(t)=\frac{sen(\pi t)}{\pi t}
\label{sinc}
\end{equation}
%
\subsection{Exponencial Generalizada}
Exponencial Real
\begin{equation*}
x(t)=Ce^{^{rt}}
\end{equation*}

En donde $C$ y $r$ son reales:
\begin{itemize}
\item Exponencial creciente: \quad $C$ es constante y $r >0$
\item  Exponencial decreciente: $C$ es constante y $r <0$
\item Constante: \qquad \qquad \qquad$C$ es constante y $r =0$
\end{itemize}
Exponencial Compleja
\begin{itemize}
\item Exponencial compleja: \quad$C$ es constante y $r=j\omega$
\item Exponencial defasada: \quad$C=C_{1}e^{j\theta}$ y $r =j\omega$
\item Exponencial defasada creciente o decreciente: \quad $C=C_{1}e^{j\theta}$ y $r =r_{1}+j\omega$
\end{itemize}

%
\section{Señales periódicas en tiempo continuo}
Una señal es periódica si se repite en intervalos regules, es decir, si cumple con
\[x(t)=x(t+mT)  \qquad m \: entero\]

en donde
\[T=\frac{1}{f}\\  \]
\[\omega=\frac{2\pi}{T}  \]

siendo \\
$T$ el periodo de la señal\\
$f$ la frecuencia de la señal\\
$\omega$ la frecuencia angular de la señal

%
%Tren de impulsos $comb(t)$
\vspace*{1cm}
\subsection{Ejemplo de convolución en tiempo discreto}
\[
x[n]=\delta[n]+3\delta[n-3]
\]
\[
h[n]=u[n]-u[n-3]
\]
Aplicando la propiedad de asociatividad
\[
y[n]=x[n]*h[n]=\delta[n]*h[n]+\delta[n-3]*h[n]
\]
Aplicando la propiedad de la convolución con la muestra unitaria
\[
y[n]=h[n]+h[n-3]=u[n]-u[n-3]+u[n-3]-u[n-6]
\]
\begin{lstlisting}[frame=single]
>> t=-3:.01:5;
>> plot(t,s1((2*t)+1))
>> grid
>> axis([-3 5 -.5 3.5 ])
\end{lstlisting}
\end{document}
