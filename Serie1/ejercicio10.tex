\documentclass[a4paper,openright,12pt]{report}
\usepackage[spanish]{babel}
\usepackage[utf8]{inputenc}
\usepackage{amssymb, amsmath, amsbsy} % simbolitos
\usepackage{upgreek} % para poner letras griegas sin cursiva
\usepackage{cancel} % para tachar
\usepackage{mathdots} % para el comando \iddots
\usepackage{mathrsfs} % para formato de letra
\usepackage{stackrel} % para el comando \stackbin
\usepackage{color}

\begin{document}
Ejercicio 10\\

Conexión de sistemas en serie o cascada\\

\begin{equation}
S_{1}: y_{1}\left [ n \right ]=nx_{1}\left [ n \right ]
\end{equation}
\begin{equation}
S_{2}: y_{2}\left [ n \right ]=x_{2}\left [ -n-1 \right ]
\end{equation}\\

S:\\

\begin{equation}
x_{2}\left [ n \right ]=y_{1}\left [ n \right ]=nx_{1}\left [ n \right ]
\end{equation}
\begin{equation}
x_{2}\left [ -n-1 \right ]=nx_{1}\left [ -n-1 \right ]
\end{equation}\\

\begin{equation}
y\left [ n \right ]=y_{2}\left [ n \right ]=nx_{1}\left [ -n-1 \right ]
\end{equation}
\begin{equation}
y\left [ n \right ]=nx_{1}\left [ -n-1 \right ]
\end{equation}\\

Ahora sacaremos su sistema invertido\\
\begin{equation}
x_{1}\left [ n \right ]=y_{2}\left [ n \right ]=x_{2}\left [ -n-1 \right ]
\end{equation}
\begin{equation}
nx_{1}\left [ n \right ]=nx_{2}\left [ -n-1 \right ]
\end{equation}
\begin{equation}
x\left [ n \right ]=nx_{2}\left [ -n-1 \right ]
\end{equation}\\

Vemos que el sistema normal e invertido da el mismo resultado.








\end{document}