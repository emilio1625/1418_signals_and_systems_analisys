\documentclass[a4paper,openright,12pt]{report}
\usepackage[spanish]{babel}
\usepackage[utf8]{inputenc}
\usepackage{amssymb, amsmath, amsbsy} % simbolitos
\usepackage{upgreek} % para poner letras griegas sin cursiva
\usepackage{cancel} % para tachar
\usepackage{mathdots} % para el comando \iddots
\usepackage{mathrsfs} % para formato de letra
\usepackage{stackrel} % para el comando \stackbin
\usepackage{color}




\begin{document}
Ejercicio 5\\

Evalúe las siguientes integrales:
\begin{equation}
y(t)=\int_{-\infty }^{\infty }[3\delta (t)+e^{-(t-1)}\delta (t)+cos(2\pi t)\dot{\delta} (t)+e^{-t}\ddot{\delta }(t) ]dt
\end{equation}

\begin{equation}
y(t)=\int_{0}^{\infty }e^{-(t-1)}\delta (t+10)dt
\end{equation}\\

Para la primera ecuación tenemos que gracias a la propiedad de      linealidad podemos realizar esta descomposición:\\

\begin{equation}
y(t)=3\int_{-\infty }^{\infty }\delta (t)dt+\int_{-\infty }^{\infty }e^{-(t-1)}\delta (t)dt+\int_{-\infty }^{\infty }cos(2\pi t)\dot{\delta (t)}dt+\int_{-\infty }^{\infty }e^{-t}\ddot{\delta}(t)dt
\end{equation}\\
Sabiendo que la transformada de Laplace de delta de Dirac es:
\begin{equation}
L\left \{ \delta \left ( t \right ) \right \}=1 
\end{equation}\\
Y que:
\begin{equation}
L\left \{ f(t)\delta (t-a) \right \}=\int_{0 }^{\infty }e^{-st}f(t)\delta (t-a)dt=e^{-as}f(a)
\end{equation}\\
Basándonos en esos conceptos ya nos es posible operar\\
Utilizando la transformada de Laplace
\begin{equation}
3L\left \{ \delta (t) \right \}=3\ast 1=3
\end{equation}\\
Aplicando la transformada al otro termino quedaría:
\begin{equation}
L\left \{ e^{-(t-1)}\delta (t) \right \}
\end{equation}\\
Ayudándonos de la ecuación numero 5 tenemos que a=0 por lo tanto:\\
\begin{equation}
e^{-as}\cdot f(a)=e^{-0s}\cdot e^{-(0-1)}=e^{o}\cdot e^{1}=1\cdot e=e
\end{equation}\\

Pasando al otro termino de la ecuación tendríamos:
\begin{equation}
L\left \{ cos(2\pi t)\dot{\delta }(t) \right \}
\end{equation}\\
La transformada de la derivada de la función impulso va a ser cero:
\begin{equation}
L\left \{ \dot{\delta }(t) \right \}=0
\end{equation}\\
Aplicando la ecuación numero 5 obtendremos:
\begin{equation}
0\cdot cos(2\pi (0))=0\cdot 1=0
\end{equation}\\
Para nuestro ultimo termino.\\
La tranformada de la segunda derivada de la función impulso es cero:
\begin{equation}
L\left \{ \ddot{\delta }(t) \right \}=0
\end{equation}\\
Evaluando con base a la ecuación 5 tendremos:
\begin{equation}
0\cdot e^{-0}=0\cdot 1=0
\end{equation}\\
Por ultimo sumamos nuestros resultados y nos quedaría:
\begin{equation}
3+e+0+0=3+e
\end{equation}\\
Finalmente nuestro resultado es:
\begin{equation}
\underline{3+e}
\end{equation}\\

Para nuestra otra función integral tendremos que a=10:
\begin{equation}
\left [ e^{-10}\cdot e^{-(10-1)} \right ]=e^{-10}\cdot e^{-10}\approx 0.00000000206= 0
\end{equation}\\
Finalmente este es nuestro resultado:
\begin{equation}
\underline{0}
\end{equation}\\




\end{document}